\chapter{Global Declarations\label{chap:GlobalDeclarations}}
\hypertarget{def-globaldeclarationterm}{}
Global declarations are grammatically derived from $\Ndecl$ and represented as ASTs by $\decl$.

There are four kinds of global declarations:
\begin{itemize}
  \item Subprogram declarations, defined in \chapref{SubprogramDeclarations};
  \item Type declarations, defined in \chapref{TypeDeclarations};
  \item Global storage declarations, defined in \chapref{GlobalStorageDeclarations};
  \item Global pragmas.
\end{itemize}

The typing of global declarations is defined in \secref{GlobalDeclarationsTyping}.
As the only kind of global declarations that are associated with semantics are global storage declarations,
their semantics is given in \secref{GlobalStorageDeclarationsSemantics}.

Global pragmas are statically checked by the typechecker, but do not produce type AST nodes,
and thus are not associated with a dynamic semantics.

%%%%%%%%%%%%%%%%%%%%%%%%%%%%%%%%%%%%%%%%%%%%%%%%%%%%%%%%%%%%%%%%%%%%%%%%%%%%%%%%
\section{Syntax}
%%%%%%%%%%%%%%%%%%%%%%%%%%%%%%%%%%%%%%%%%%%%%%%%%%%%%%%%%%%%%%%%%%%%%%%%%%%%%%%%
Subprogram declarations:
\begin{flalign*}
\Ndecl  \derives \ & \Tfunc \parsesep \Tidentifier \parsesep \Nparamsopt \parsesep \Nfuncargs \parsesep \Nreturntype \parsesep \Nfuncbody &\\
|\ & \Tfunc \parsesep \Tidentifier \parsesep \Nparamsopt \parsesep \Nfuncargs \parsesep \Nfuncbody &\\
|\ & \Taccessor \parsesep \Tidentifier \parsesep \Nparamsopt \parsesep \Nfuncargs \parsesep \Tbiarrow \parsesep \Nty &\\
   & \wrappedline\ \Tbegin \parsesep \Naccessors \parsesep \Tend \parsesep \Tsemicolon &\\
\end{flalign*}

Type declarations:
\begin{flalign*}
\Ndecl  \derives \ & \Ttype \parsesep \Tidentifier \parsesep \Tof \parsesep \Ntydecl \parsesep \Nsubtypeopt \parsesep \Tsemicolon&\\
|\ & \Ttype \parsesep \Tidentifier \parsesep \Nsubtype \parsesep \Tsemicolon&\\
\end{flalign*}

Global storage declarations:
\begin{flalign*}
\Ndecl  \derives \ & \Nglobaldeclkeyword \parsesep \Nignoredoridentifier \parsesep &\\
    & \wrappedline\ \option{\Tcolon \parsesep \Nty} \parsesep \Teq \Nexpr \parsesep \Tsemicolon &\\
|\ & \Tconfig \parsesep \Nignoredoridentifier \parsesep \Tcolon \parsesep \Nty \parsesep \Teq \parsesep \Nexpr \parsesep \Tsemicolon &\\
|\ & \Tvar \parsesep \Nignoredoridentifier \parsesep \Tcolon \parsesep \Nty \parsesep \Tsemicolon&\\
\end{flalign*}

Pragmas:
\begin{flalign*}
\Ndecl  \derives \ & \Tpragma \parsesep \Tidentifier \parsesep \ClistZero{\Nexpr} \parsesep \Tsemicolon&
\end{flalign*}

%%%%%%%%%%%%%%%%%%%%%%%%%%%%%%%%%%%%%%%%%%%%%%%%%%%%%%%%%%%%%%%%%%%%%%%%%%%%%%%%
\section{Abstract Syntax}
%%%%%%%%%%%%%%%%%%%%%%%%%%%%%%%%%%%%%%%%%%%%%%%%%%%%%%%%%%%%%%%%%%%%%%%%%%%%%%%%
\begin{flalign*}
\decl \derives\ & \DFunc(\func) & \\
  |\ & \DGlobalStorage(\globaldecl) & \\
  |\ & \DTypeDecl(\Tidentifier, \ty, (\Tidentifier, \overtext{\Field^*}{with fields})?) & \\
  |\ & \DPragma(\Tidentifier, \overtext{\expr^*}{args}) &
\end{flalign*}

\hypertarget{build-decl}{}
The relation
\[
  \builddecl : \overname{\parsenode{\Ndecl}}{\vparsednode} \;\aslrel\; \overname{\decl^*}{\vastnode}
\]
transforms a parse node $\vparsednode$ into an AST node $\vastnode$.

The case rules for building global declarations are the following:
\begin{itemize}
  \item \ASTRuleRef{GlobalStorageDecl} for global storage declarations
  \item \ASTRuleRef{TypeDecl} for type declarations
  \item \ASTRuleRef{GlobalPragma} for global pragmas
\end{itemize}

\ASTRuleDef{GlobalPragma}
\begin{mathpar}
\inferrule[global\_pragma]{
  \buildclist[\Nexpr](\vargs) \astarrow \astversion{\vargs}
}{
  {
  \begin{array}{r}
  \builddecl(\overname{\Ndecl(\Tpragma, \Tidentifier(\id), \namednode{\vargs}{\ClistZero{\Nexpr}}, \Tsemicolon)}{\vparsednode})
  \astarrow \\
    \overname{\left[\DPragma(\id, \astversion{\vargs})\right]}{\vastnode}
  \end{array}
  }
}
\end{mathpar}

%%%%%%%%%%%%%%%%%%%%%%%%%%%%%%%%%%%%%%%%%%%%%%%%%%%%%%%%%%%%%%%%%%%%%%%%%%%%%%%%
\section{Typing Global Declarations\label{sec:GlobalDeclarationsTyping}}
%%%%%%%%%%%%%%%%%%%%%%%%%%%%%%%%%%%%%%%%%%%%%%%%%%%%%%%%%%%%%%%%%%%%%%%%%%%%%%%%
\hypertarget{def-typecheckdecl}{}
The function
\[
  \typecheckdecl(
    \overname{\globalstaticenvs}{\genv} \aslsep
    \overname{\decl}{\vd}
  )
  \aslto (\overname{\decl}{\newd} \times \overname{\globalstaticenvs}{\newgenv})
  \cup \overname{\TTypeError}{\TypeErrorConfig}
\]
annotates a global declaration $\vd$ in the global static environment $\genv$,
yielding an annotated global declaration $\newd$ and modified global static environment $\newgenv$.
\ProseOtherwiseTypeError

\TypingRuleDef{TypecheckDecl}
\ProseParagraph
\OneApplies
\begin{itemize}
  \item \AllApplyCase{func}
  \begin{itemize}
    \item $\vd$ is a subprogram AST node with a subprogram definition $\vf$, that is, $\DFunc(\vf)$;
    \item annotating and declaring the subprogram for $\vf$ in $\genv$ as per \\
          \TypingRuleRef{AnnotateAndDeclareFunc}
          yields the environment $\tenvone$ and a subprogram definition $\vfone$\ProseOrTypeError;
    \item \Proseannotatesubprogram{$\vfone$}{$\tenv$}{$\vftwo$\ProseOrTypeError};
    \item applying $\addsubprogram$ to bind $\vftwo.\funcname$ to $\vftwo$ in $\tenvone$ yields $\newtenv$;
    \item define $\newd$ as the subprogram AST node with $\vftwo$, that is, $\DFunc(\vftwo)$;
    \item define $\newgenv$ as the global component of $\newtenv$.
  \end{itemize}

  \item \AllApplyCase{global\_storage}
  \begin{itemize}
    \item $\vd$ is a global storage declaration with description $\gsd$, that is, \\ $\DGlobalStorage(\gsd)$;
    \item declaring the global storage with description $\gsd$ in $\genv$ yields the new environment
          $\newgenv$ and new global storage description $\gsdp$\ProseOrTypeError;
    \item define $\newd$ as the global storage declaration with description $\gsdp$, that is, \\ $\DGlobalStorage(\gsdp)$.
  \end{itemize}

  \item \AllApplyCase{type}
  \begin{itemize}
    \item $\vd$ is a type declaration with identifier $\vx$, type $\tty$,
          and \optional\ field initializers $\vs$, that is, $\DTypeDecl(\vx, \tty, \vs)$;
    \item declaring the type described by $(\vx, \tty, \vs)$ in $\genv$
          as per \\
          \TypingRuleRef{DeclaredType} yields the modified global static environment \\
          $\newgenv$\ProseOrTypeError;
    \item define $\newd$ as $\vd$.
  \end{itemize}
\end{itemize}

\FormallyParagraph
\begin{mathpar}
\inferrule[func]{
  \annotateanddeclarefunc(\genv, \vf) \typearrow (\tenvone, \vfone) \OrTypeError\\\\
  \annotatesubprogram(\tenvone, \vfone) \typearrow \vftwo \OrTypeError\\\\
  \addsubprogram(\tenvone, \vftwo.\funcname, \vftwo) \typearrow \newtenv
}{
  \typecheckdecl(\genv, \overname{\DFunc(\vf)}{\vd})
  \typearrow (\overname{\DFunc(\vftwo)}{\newd}, \overname{G^\newtenv}{\newgenv})
}
\end{mathpar}

\begin{mathpar}
\inferrule[global\_storage]{
  \declareglobalstorage(\genv, \gsd) \typearrow (\newgenv, \gsdp) \OrTypeError
}{
  {
    \begin{array}{r}
  \typecheckdecl(\genv, \overname{\DGlobalStorage(\gsd)}{\vd})
  \typearrow \\ (\overname{\DGlobalStorage(\gsdp)}{\newd}, \newgenv)
    \end{array}
  }
}
\end{mathpar}

\begin{mathpar}
\inferrule[type]{
  \declaretype(\genv, \vx, \tty, \vs) \typearrow \newgenv \OrTypeError
}{
  \typecheckdecl(\genv, \overname{\DTypeDecl(\vx, \tty, \vs)}{\vd}) \typearrow (\overname{\vd}{\newd}, \newgenv)
}
\end{mathpar}
\CodeSubsection{\TypecheckDeclBegin}{\TypecheckDeclEnd}{../Typing.ml}

\TypingRuleDef{Subprogram}
\hypertarget{def-annotatesubprogram}{}
The function
\[
\begin{array}{r}
  \annotatesubprogram(\overname{\staticenvs}{\tenv} \aslsep \overname{\func}{\vf} \aslsep \overname{\TSideEffectSet}{\vsesfuncsig})
  \aslto \\
  (\overname{\func}{\vfp} \times \overname{\TSideEffectSet}{\vses})\ \cup \overname{\TTypeError}{\TypeErrorConfig}
\end{array}
\]
annotates a subprogram $\vf$ in an environment $\tenv$ and \sideeffectsetterm\ $\vsesfuncsig$, resulting in an annotated subprogram $\vfp$
and inferred \sideeffectsetterm\ $\vses$.
\ProseOtherwiseTypeError

Note that the return type in $\vf$ has already been annotated by $\annotatefuncsig$.

\ProseParagraph
\AllApply
\begin{itemize}
  \item annotating $\vf.\funcbody$ in $\tenv$ as per \TypingRuleRef{Block} yields $(\newbody, \vsesbody)$\ProseOrTypeError;
  \item \OneApplies
  \begin{itemize}
    \item \AllApplyCase{procedure}
    \begin{itemize}
      \item $\vf.\funcreturntype$ is $\None$;
    \end{itemize}

    \item \AllApplyCase{function}
    \begin{itemize}
      \item $\vf.\funcreturntype$ is not $\None$;
      \item applying $\checkstmtreturnsorthrows$ to $\newbody$ yields $\True$\ProseOrTypeError;
    \end{itemize}
  \end{itemize}
  \item $\vfp$ is $\vf$ with the subprogram body substituted with $\newbody$;
  \item \Proseeqdef{$\vses$}{the union of $\vsesfuncsig$ and $\vsesbody$ with every instance of a
        \ReadLocalTerm\ or a \WriteLocalTerm\ removed}.
\end{itemize}

\FormallyParagraph
\begin{mathpar}
\inferrule[procedure]{
  \annotateblock{\tenv, \vf.\funcbody} \typearrow (\newbody, \vsesbody) \OrTypeError\\\\
  \commonprefixline\\\\
  \vf.\funcreturntype = \None\\\\
  \commonsuffixline\\\\
  \vfp \eqdef \substrecordfield(\vf, \funcbody, \newbody)\\
  \vses \eqdef \vsesfuncsig \cup (\vsesbody \setminus \{ \vs \;|\; \configdomain{\vs} \in \{\ReadLocal, \WriteLocal\} \})
}{
  \annotatesubprogram(\tenv, \vf, \vsesfuncsig) \typearrow \vfp
}
\end{mathpar}

\begin{mathpar}
\inferrule[function]{
  \annotateblock{\tenv, \vf.\funcbody} \typearrow (\newbody, \vsesbody) \OrTypeError\\\\
  \commonprefixline\\\\
  \vf.\funcreturntype \neq \None\\
  \checkstmtreturnsorthrows(\newbody) \typearrow \True \OrTypeError\\\\
  \commonsuffixline\\\\
  \vfp \eqdef \substrecordfield(\vf, \funcbody, \newbody)\\
  \vses \eqdef \vsesfuncsig \cup (\vsesbody \setminus \{ \vs \;|\; \configdomain{\vs} \in \{\ReadLocal, \WriteLocal\} \})
}{
  \annotatesubprogram(\tenv, \vf, \vsesfuncsig) \typearrow \vfp
}
\end{mathpar}
\identi{GHGK} \identr{HWTV} \identr{SCHV} \identr{VDPC}
\identr{TJKQ} \identi{LFJZ} \identi{BZVB} \identi{RQQB}
\CodeSubsection{\SubprogramBegin}{\SubprogramEnd}{../Typing.ml}

\TypingRuleDef{CheckStmtReturnsOrThrows}
\hypertarget{def-checkstmtreturnsorthrows}{}
The helper function
\[
  \checkstmtreturnsorthrows(\overname{\stmt}{\vs})
  \typearrow \{\True\} \cup \overname{\TTypeError}{\TypeErrorConfig}
\]
checks whether all control-flow paths defined by the statement $\vs$ terminate by either
a statement returning a value, a \texttt{throw} statement, or the \texttt{Unreachable()} statement.

\ProseParagraph
\AllApply
\begin{itemize}
  \item applying $\controlflowfromstmt$ to $\vs$ yields a \controlflowsymbolterm\ $\vctrlflow$;
  \item checking that $\vctrlflow$ is different from $\maynotinterrupt$ yields $\True$\ProseTerminateAs{\BadSubprogramDeclaration};
  \item the result is $\True$.
\end{itemize}

\FormallyParagraph
\begin{mathpar}
\inferrule{
  \controlflowfromstmt(\vs) \typearrow \vctrlflow\\
  \checktrans{\vctrlflow \neq \maynotinterrupt}{\BadSubprogramDeclaration} \typearrow \True \OrTypeError
}{
  \checkstmtreturnsorthrows(\vs) \typearrow \True
}
\end{mathpar}

\TypingRuleDef{ControlFlowFromStmt}
\hypertarget{def-controlflowsymbolterm}{}
\hypertarget{def-controlflowstate}{}
We define \controlflowsymbolsterm\ as follows:
\[
  \controlflowstate \eqdef \{\assertednotinterrupt, \interrupt, \maynotinterrupt\}
\]
\hypertarget{def-controlflowleq}{}
\controlflowsymbolsterm\ are totally ordered via the relation $\controlflowleq$, defined as follows:
\[
  \assertednotinterrupt\ \controlflowleq \interrupt \controlflowleq \maynotinterrupt \enspace.
\]

\hypertarget{def-controlflowfromstmt}{}
The helper function
\[
  \controlflowfromstmt(\overname{\stmt}{\vs})
  \typearrow \overname{\controlflowstate}{\vctrlflow}
\]
statically analyzes the statement $\vs$
and determines the \controlflowsymbolterm\ $\vctrlflow$ to be one of the following:
\hypertarget{def-assertednotinterrupt}{}
\begin{description}
  \item[$\assertednotinterrupt$] evaluating $\vs$ in any environment will evaluate \texttt{Unreachable()};
  \hypertarget{def-interrupt}{}
  \item[$\interrupt$] evaluating $\vs$ in any environment will \underline{always} end by either evaluating
      a return statement with an expression,
      or evaluating a \texttt{throw} statement;
  \hypertarget{def-maynotinterrupt}{}
  \item[$\maynotinterrupt$] evaluating $\vs$ in any environment might not end by evaluating
      a \texttt{return} statement with an expression or a \texttt{throw} statement.
\end{description}

\ProseParagraph
\OneApplies
\begin{itemize}
  \item \AllApplyCase{falls\_through}
  \begin{itemize}
    \item the AST label of $\vs$ is $\SPass$, $\SDecl$, $\SAssign$, $\SAssert$, $\SCall$, $\SPrint$ or $\SPragma$;
    \item $\vctrlflow$ is $\maynotinterrupt$;
  \end{itemize}

  \item \AllApplyCase{unreachable}
  \begin{itemize}
    \item $\vs$ is $\SUnreachable$;
    \item $\vctrlflow$ is $\assertednotinterrupt$;
  \end{itemize}

  \item \AllApplyCase{return\_throw}
  \begin{itemize}
    \item the AST label of $\vs$ is either $\SReturn$ or $\SThrow$;
    \item $\vctrlflow$ is $\interrupt$;
  \end{itemize}

  \item \AllApplyCase{s\_seq}
  \begin{itemize}
    \item $\vs$ is the \sequencingstatementterm\ for $\vsone$ and $\vstwo$;
    \item applying $\controlflowfromstmt$ to $\vsone$ yields $\vctrlflowone$;
    \item applying $\controlflowfromstmt$ to $\vstwo$ yields $\vctrlflowtwo$;
    \item applying $\controlflowseq$ to $\vctrlflowone$ and $\vctrlflowtwo$ yields $\vctrlflow$.
  \end{itemize}

  \item \AllApplyCase{s\_cond}
  \begin{itemize}
    \item $\vs$ is the \conditionalstatementterm\ for an expression and statements $\vsone$ and $\vstwo$;
    \item applying $\controlflowfromstmt$ to $\vsone$ yields $\vctrlflowone$;
    \item applying $\controlflowfromstmt$ to $\vstwo$ yields $\vctrlflowtwo$;
    \item applying $\controlflowjoin$ to $\vctrlflowone$ and $\vctrlflowtwo$ yields $\vctrlflow$.
  \end{itemize}

  \item \AllApplyCase{s\_while\_for}
  \begin{itemize}
    \item $\vs$ is either a \whilestatementterm\ or a \forstatementterm;
    \item define $\vctrlflow$ as $\maynotinterrupt$.
  \end{itemize}

  \item \AllApplyCase{s\_repeat}
  \begin{itemize}
    \item $\vs$ is the \repeatstatementterm\ with the body statement $\vbody$;
    \item applying $\controlflowfromstmt$ to $\vbody$ yields $\vctrlflow$.
  \end{itemize}

  \item \AllApplyCase{s\_try}
  \begin{itemize}
    \item $\vs$ is the \trystatement{$\vbody$}{\\ $\catchers$}{$\votherwiseopt$};
    \item applying $\controlflowfromstmt$ to $\vbody$ yields $\vreszero$;
    \item \Proseeqdef{$\vresone$}{
          the application of $\controlflowjoin$ to $\controlflowfromstmt(\vo)$
          and $\vreszero$, if $\votherwiseopt = \langle\vo\rangle$,
          and $\vreszero$, otherwise};
    \item for each catcher in $\catchers$ associated with a statement $\vs$,
          applying \\
          $\controlflowfromstmt$ to $\vs$ yields $\vcf_\vs$;
    \item \Proseeqdef{$\vctrlflow$}{the application of $\controlflowjoin$ to $\vresone$,
          and $\vcf_\vs$, for each catcher in $\catchers$ associated with a statement $\vs$}.
  \end{itemize}
\end{itemize}

\FormallyParagraph
\begin{mathpar}
\inferrule[falls\_through]{
  \astlabel(\vs) \in \{\SPass, \SDecl, \SAssign, \SAssert, \SCall, \SPrint, \SPragma\}
}{
  \controlflowfromstmt(\vs) \typearrow \overname{\maynotinterrupt}{\vctrlflow}
}
\end{mathpar}

\begin{mathpar}
\inferrule[unreachable]{}{
  \controlflowfromstmt(\overname{\SUnreachable}{\vs}) \typearrow \overname{\assertednotinterrupt}{\vctrlflow}
}
\end{mathpar}

\begin{mathpar}
\inferrule[return\_throw]{
  \astlabel(\vs) \in \{\SReturn, \SThrow\}
}{
  \controlflowfromstmt(\vs) \typearrow \overname{\interrupt}{\vctrlflow}
}
\end{mathpar}

\begin{mathpar}
\inferrule[s\_seq]{
  \controlflowfromstmt(\vsone) \typearrow \vctrlflowone\\
  \controlflowfromstmt(\vstwo) \typearrow \vctrlflowtwo\\
  \controlflowseq(\vctrlflowone, \vctrlflowtwo) \typearrow \vctrlflow
}{
  \controlflowfromstmt(\overname{\SSeq(\vsone, \vstwo)}{\vs}) \typearrow \vctrlflow
}
\end{mathpar}

\begin{mathpar}
\inferrule[s\_cond]{
  \controlflowfromstmt(\vsone) \typearrow \vctrlflowone\\
  \controlflowfromstmt(\vstwo) \typearrow \vctrlflowtwo\\
  \controlflowjoin(\{\vctrlflowone, \vctrlflowtwo\}) \typearrow \vctrlflow
}{
  \controlflowfromstmt(\overname{\SCond(\Ignore, \vsone, \vstwo)}{\vs}) \typearrow \vctrlflow
}
\end{mathpar}

The reasoning for the next case is as follows:
\begin{itemize}
  \item Conservatively, that is, without attempting to reason about the condition expressions,
    a $\SWhile$ loop, and a $\SFor$ are like a conditional statement that either executes
    $\SPass$ (when the loop condition does not hold) or executes the body and then loops back.
  \item The control flow state for $\SPass$ is $\maynotinterrupt$.
  \item The overall control flow state is then $\controlflowjoin(\{\maynotinterrupt, \Ignore\})$,
    which is always $\maynotinterrupt$, since $\maynotinterrupt$ is the maximal element,
    with respect to $\controlflowleq$.
\end{itemize}

\begin{mathpar}
\inferrule[s\_while\_for]{
  \astlabel(\vs) \in \{\SWhile, \SFor\}
}{
  \controlflowfromstmt(\vs) \typearrow \overname{\maynotinterrupt}{\vctrlflow}
}
\end{mathpar}

The reasoning for the next case is as follows:
\begin{itemize}
  \item A statement $\SRepeat(\vbody, \vcond, \Ignore)$ is equivalent (ignoring limits) to \\
$\SSeq(\vbody, \SWhile(\vcond, \Ignore, \vbody))$;
  \item Let us denote $\controlflowfromstmt(\vbody) \typearrow \vctrlflow$.
  Observe that by case \textsc{s\_while\_for} above, we have that \\
  $\controlflowfromstmt(\SWhile(\vcond, \Ignore, \vbody)) \typearrow \maynotinterrupt$
  holds;
  \item Applying the rule for the \textsc{s\_seq} case above, requires applying $\controlflowseq$
  to $\vctrlflow$ and $\maynotinterrupt$, which always yields $\vctrlflow$
  (to see this, consider the case $\vctrlflow=\maynotinterrupt$
  and the case \\ $\vctrlflow\neq\maynotinterrupt$).
\end{itemize}

\begin{mathpar}
\inferrule[s\_repeat]{
  \controlflowfromstmt(\vbody) \typearrow \vctrlflow
}{
  \controlflowfromstmt(\overname{\SRepeat(\vbody, \Ignore, \Ignore)}{\vs}) \typearrow \vctrlflow
}
\end{mathpar}

The rule for \textsc{s\_try} conservatively approximates the results from all control flows
passing through the statement by returning the maximal $\controlflowsymbolterm$,
with respect to $\controlflowleq$,
computed for each control flow path.
\begin{mathpar}
\inferrule[s\_try]{
  \controlflowfromstmt(\vbody) \typearrow \vreszero\\
  {
  \vresone \eqdef
  \begin{cases}
    \controlflowjoin(\{\controlflowfromstmt(\vo), \vreszero\}) & \text{if }\votherwiseopt = \langle\vo\rangle\\
    \vreszero & \text{otherwise}
  \end{cases}
  }\\
  (\Ignore,\Ignore,\vs)\in\catchers: \controlflowfromstmt(\vs) \typearrow \vcf_\vs\\
  \vctrlflow \eqdef \controlflowjoin(\{(\Ignore,\Ignore,\vs)\in\catchers: \vcf_\vs\} \cup \{\vresone\})
}{
  \controlflowfromstmt(\overname{\STry(\vbody, \catchers, \votherwiseopt)}{\vs}) \typearrow \vctrlflow
}
\end{mathpar}

\TypingRuleDef{ControlFlowSeq}
\hypertarget{def-controlflowseq}{}
The helper function
\[
\controlflowseq(\overname{\controlflowstate}{\vtone} \aslsep \overname{\controlflowstate}{\vttwo})
\aslto \overname{\controlflowstate}{\vctrlflow}
\]
combines two \controlflowsymbolterm{s} considering them as part of a control flow path where the analysis of the
path prefix yields $\vtone$ and the analysis of the path suffix yields $\vttwo$,
into a single \controlflowsymbolterm\ $\vctrlflow$.

\ProseParagraph
Define $\vctrlflow$ as $\vttwo$ if $\vtone$ is $\maynotinterrupt$ and $\vtone$, otherwise.

\FormallyParagraph
\begin{mathpar}
\inferrule{
  \vctrlflow \eqdef \choice{\vtone = \maynotinterrupt}{\vttwo}{\vtone}
}{
  \controlflowseq(\vtone, \vttwo) \typearrow \vctrlflow
}
\end{mathpar}

\TypingRuleDef{ControlFlowJoin}
\hypertarget{def-controlflowjoin}{}
The helper function
The helper function
\[
\controlflowjoin(\overname{\pow{\controlflowstate}}{\vs})
\aslto \overname{\controlflowstate}{\vctrlflow}
\]
returns the maximal element in the set of \controlflowsymbolsterm\ $\vs$,
with respect to $\controlflowleq$ in $\vctrlflow$.

\ProseParagraph
\Proseeqdef{$\vctrlflow$}{the maximal element in the set of \controlflowsymbolsterm\ $\vs$,
with respect to $\controlflowleq$}.

\FormallyParagraph
\begin{mathpar}
\inferrule{}{
  \controlflowjoin(\vs) \typearrow \overname{\max_{\controlflowleq}(\vs)}{\vctrlflow}
}
\end{mathpar}
